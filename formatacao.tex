\usepackage[utf8]{inputenc}
\usepackage[T1]{fontenc}
\usepackage{amsmath}
\usepackage{amssymb,amsfonts,textcomp}
\usepackage{color}
\usepackage{array}
\usepackage{supertabular}
\usepackage{listings}         % Para as linguagens de programação
\usepackage{lastpage}		  % Usado pela Ficha catalográfica
\usepackage{indentfirst}	  % Indenta o primeiro parágrafo de cada seção.
\usepackage{hhline}
\usepackage{hyperref}
% Informações do PDF
\hypersetup{ colorlinks=false, linkcolor=blue, citecolor=blue, filecolor=blue, urlcolor=blue, pdftitle=DEPARTAMENTO DE CIÊNCIA DA COMPUTAÇÃO - UFRJ, pdfauthor=, pdfsubject=, pdfkeywords=}

\usepackage[pdftex]{graphicx}
\graphicspath{ {./figuras/} }
% ---
% Pacotes glossaries
% ---
\usepackage[subentrycounter,seeautonumberlist,nonumberlist=true]{glossaries}
% para usar o xindy ao invés do makeindex:
%\usepackage[xindy={language=portuguese},subentrycounter,seeautonumberlist,nonumberlist=true]{glossaries}
% ---
% Citações de referências no formato alfabético e negrito
\usepackage[alf, abnt-emphasize=bf]{abntex2cite}
% Margens definidas em 25 mm para uso como documento em PDF
% Para imprimir use as seguintes margens:
% \usepackage[left=30mm, top=30mm, right=20 mm, bottom=20mm] {geometry}

\usepackage[margin=25 mm]{geometry}

%%%%%%%%%%%%%%%%%%%%%%%%%%%%%%%%%%%%%%%%%%%%%%%%%%%%%%%%%%%%%
% Definições de Macros utilizadas no TCC
%%%%%%%%%%%%%%%%%%%%%%%%%%%%%%%%%%%%%%%%%%%%%%%%%%%%%%%%%%%%%
\instituicao{UNIVERSIDADE FEDERAL DO RIO DE JANEIRO
	\par
	INSTITUTO DE MATEMÁTICA
	\par
	CURSO DE BACHARELADO EM CIÊNCIA DA COMPUTAÇÃO}
\title{UNIVERSE \\ Uma \textit{framework} para processamento distribuído}
\author{Ericson José da Silva Soares \\ Raphael de Carvalho Almeida \\ Vitor Augusto da Silva Vasconcellos}
\autor{Ericson José da Silva Soares \\ Raphael de Carvalho Almeida \\ Vitor Augusto da Silva Vasconcellos}
\orientador{Prof. Gabriel Pereira Silva}
\coorientador{Prof. Claudio Miceli Farias}
\local{RIO DE JANEIRO}
\data{2019}
\preambulo{Trabalho de conclusão de curso de graduação apresentado ao Departamento de Ciência da Computação da Universidade Federal do Rio de Janeiro como parte dos requisitos para obtenção do grau de Bacharel em Ciência da Computação.}

\usepackage{tocloft}

\usepackage{pdfpages}

%%%%%%%%%%%%%%%%%%%%%%%%%%%%%%%%%%%%%%%%%%%%%%%%%%%%%%%%%%%
% Corrige a fonte dos capítulos, seções, resumos, etc.
%%%%%%%%%%%%%%%%%%%%%%%%%%%%%%%%%%%%%%%%%%%%%%%%%%%%%%%%%%%
\renewcommand{\ABNTEXchapterfont}{\bfseries \rmfamily}
\renewcommand{\ABNTEXchapterfontsize}{\normalsize}

\renewcommand{\ABNTEXsectionfont}{\mdseries}
\renewcommand{\ABNTEXsectionfontsize}{\normalsize}

\renewcommand{\ABNTEXsubsectionfont}{\bfseries}
\renewcommand{\ABNTEXsubsectionfontsize}{\normalsize}

\renewcommand{\lstlistingname}{Código}
\renewcommand{\lstlistlistingname}{Lista de \lstlistingname s}

%%%%%%%%%%%%%%%%%%%%%%%%%%%%%%%%%%%%%%%%%%%%%%%%%%%%%%%%%%%
% Criação de quadros com numeração
%%%%%%%%%%%%%%%%%%%%%%%%%%%%%%%%%%%%%%%%%%%%%%%%%%%%%%%%%%%
\newcommand{\quadroname}{Quadro}
\newcommand{\listofquadrosname}{Lista de quadros}

\newfloat[chapter]{quadro}{loq}{\quadroname}
\newlistof{listofquadros}{loq}{\listofquadrosname}
\newlistentry{quadro}{loq}{0}

% configurações para atender às regras da ABNT
\setfloatadjustment{quadro}{\centering}
\counterwithout{quadro}{chapter}
\renewcommand{\cftquadroname}{\quadroname\space}
\renewcommand*{\cftquadroaftersnum}{\hfill--\hfill}

% Configuração de posicionamento padrão:
\setfloatlocations{quadro}{hbtp}

% Para ajudar nas tabelas e quadros
\makeatletter
\newcommand\arraybslash{\let\\\@arraycr}
\makeatother

%%%%%%%%%%%%%%%%%%%%%%%%%%%%%%%%%%%%%%%%%%%%%%%%%%%%%%%
% Redefine a macro para imprimir a capa
%%%%%%%%%%%%%%%%%%%%%%%%%%%%%%%%%%%%%%%%%%%%%%%%%%%%%%%
\renewcommand{\imprimircapa}{%
	\begin{capa}%
		\center
		\imprimirinstituicao
		\par
		\vspace*{1cm}
		\MakeUppercase{\imprimirautor}
		\vfill
		\begin{center}
			\imprimirtitulo
		\end{center}
		\vfill
		\imprimirlocal
		\par
		\imprimirdata
		\vspace*{1cm}
	\end{capa}
}
%%%%%%%%%%%%%%%%%%%%%%%%%%%%%%%%%%%%%%%%%%%%%%%%%%%%%%%
% Redefine a macro para imprimir a folho de rosto
%%%%%%%%%%%%%%%%%%%%%%%%%%%%%%%%%%%%%%%%%%%%%%%%%%%%%%%
\renewcommand{\imprimirfolhaderosto}{%
	\begin{capa}%
		\center
		\par
		\vspace*{1cm}
		\MakeUppercase{\imprimirautor}
		\vfill
		\begin{center}
			\imprimirtitulo
		\end{center}
		\vfill
		\hspace*{\fill}\parbox[b]{.5\textwidth}{%
			\linespread{1}\selectfont
			\imprimirpreambulo
		}
		\vfill
		\flushright
		Orientador: \imprimirorientador\\

		Co-orientador: \imprimircoorientador\\
		\vfill
		\begin{center}
			\large\imprimirlocal
			\par
			\large\imprimirdata
			\vspace*{1cm}
		\end{center}
	\end{capa}
}

%%%%%%%%%%%%%%%%%%%%%%%%%%%%%%%%%%%%%%%%%%%%%%%%%%%%
% Definição das Linguagens de Programação
%%%%%%%%%%%%%%%%%%%%%%%%%%%%%%%%%%%%%%%%%%%%%%%%%%%%
\definecolor{dkgreen}{rgb}{0,0.6,0}
\definecolor{gray}{rgb}{0.5,0.5,0.5}
\definecolor{purple}{rgb}{0.8,0,0.3}
\definecolor{orange}{rgb}{1,0.4,0}
\definecolor{lightlightgray}{rgb}{.95,.95,.95}
\definecolor{lightgray}{rgb}{.9,.9,.9}
\definecolor{lightgray2}{rgb}{.85,.85,.85}
\definecolor{darkgray}{rgb}{.4,.4,.4}

% Comando para inserir a listagem de código, tem 4 parâmetros
% Linguagem, Caption, Label, Nome do Arquivo
\newcommand{\includecode}[4][C]{\mbox{\lstinputlisting[caption=#2, label=#3, escapechar=, style=custom#1]{#4}}}


%Define características em comum
\lstset{framexleftmargin=5mm,  frame=shadowbox, rulesepcolor=\color{gray}}

% Aqui você pode customizar a linguagem
\lstdefinestyle{customC}{
	language = C,
	breaklines=true,
	basicstyle=\footnotesize\ttfamily,
	keywordstyle=\bfseries\color{blue},
	commentstyle=\itshape\color{purple},
	identifierstyle=\color{black},
	stringstyle=\color{orange},
	showstringspaces=false}

% Aqui você pode customizar a linguagem
\lstdefinestyle{customJava}{
	language = Java,
	breaklines=true,
	basicstyle=\footnotesize\ttfamily,
	keywordstyle=\bfseries\color{blue},
	commentstyle=\itshape\color{purple},
	identifierstyle=\color{black},
	stringstyle=\color{orange},
	showstringspaces=false}

% Aqui você pode customizar a linguagem
\lstdefinestyle{customC++}{
	language = C++,
	breaklines=true,
	basicstyle=\footnotesize\ttfamily,
	keywordstyle=\bfseries\color{blue},
	commentstyle=\itshape\color{purple},
	identifierstyle=\color{black},
	stringstyle=\color{orange},
	showstringspaces=false}

\lstdefinestyle{customJavaScript}{
	language = JavaScript,
	breaklines=true,
	basicstyle=\footnotesize\ttfamily,
	keywordstyle=\bfseries\color{blue},
	commentstyle=\itshape\color{purple},
	identifierstyle=\color{black},
	stringstyle=\color{orange},
	showstringspaces=false}

\lstdefinestyle{customJSON}{
	language = JSON,
	breaklines=true,
	basicstyle=\footnotesize\ttfamily,
	keywordstyle=\bfseries\color{blue},
	commentstyle=\itshape\color{purple},
	identifierstyle=\color{black},
	stringstyle=\color{orange},
	showstringspaces=false}

\lstdefinestyle{customSapiens}{
	language = Sapiens,
	breaklines=true,
	basicstyle=\footnotesize\ttfamily,
	keywordstyle=\bfseries\color{blue},
	commentstyle=\itshape\color{purple},
	identifierstyle=\color{black},
	stringstyle=\color{orange},
	showstringspaces=false}

% Javascript
\lstdefinelanguage{JavaScript}{
	aboveskip=15pt,
	keywords={typeof, new, true, false, catch, function, return, null, catch, switch, var, if, in, while, do, else, case, break},
	keywordstyle=\color{blue}\bfseries,
	ndkeywords={class, export, boolean, throw, implements, import, this},
	ndkeywordstyle=\color{darkgray}\bfseries,
	identifierstyle=\color{black},
	sensitive=false,
	comment=[l]{//},
	morecomment=[s]{/*}{*/},
	commentstyle=\color{purple}\ttfamily,
	stringstyle=\color{red}\ttfamily,
	morestring=[b]',
	morestring=[b]",
	rulecolor=\color{lightgray2},
	breaklines=true,
	basicstyle=\footnotesize\ttfamily,
	frame=single,
	backgroundcolor=\color{lightlightgray}
}
% JSON
\lstdefinelanguage{JSON}{
	aboveskip=15pt,
	string=[s]{"}{"},
	stringstyle=\color{blue},
	comment=[l]{:},
	commentstyle=\color{black},
	rulecolor=\color{lightgray2},
	breaklines=true,
	basicstyle=\footnotesize\ttfamily,
	frame=single,
	backgroundcolor=\color{lightlightgray}
}

% Linguagem de Montagem
\lstdefinelanguage{Sapiens}{
	aboveskip=15pt,
	keywordstyle=\color{blue}\bfseries,
	comment=[l]{;},
	commentstyle=\color{purple},
	rulecolor=\color{lightgray2},
	breaklines=true,
	basicstyle=\footnotesize\ttfamily,
	frame=single,
	backgroundcolor=\color{lightlightgray},
	language=[x86masm]Assembler,
	morekeywords={DB, DS, DW, JN, JSR, LDA, ORG, STA, STS, TRAP}
}
